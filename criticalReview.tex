\documentclass[12pt,a4paper]{article}
\usepackage[a4paper,margin=1in]{geometry}
\usepackage{csquotes}
\usepackage{lmodern}
\usepackage{graphicx}
\usepackage{booktabs}
\usepackage{hyperref}
\hypersetup{colorlinks=true, linkcolor=black, urlcolor=blue}

%---- Headers/footers (page numbers) ----%
\usepackage{fancyhdr}
\pagestyle{fancy}
\fancyhf{}                    % clear header/footer
\fancyfoot[C]{\thepage}      % page number centered in footer
\renewcommand{\headrulewidth}{0pt}
\renewcommand{\footrulewidth}{0pt}

%---- Heading formatting (Header 1 & 2) ----%
% Header 1 (Chapter-level): 14pt, bold
\makeatletter
\renewcommand\section{\@startsection{section}{1}{\z@}%
  {-3.5ex \@plus-1ex \@minus-.2ex}%
  {2.3ex \@plus.2ex}%
  {\normalfont\large\bfseries}}
\makeatother
% Header 2 (subsection): 12pt, bold
\makeatletter
\renewcommand\subsection{\@startsection{subsection}{2}{\z@}%
  {-3.25ex\@plus-1ex \@minus-.2ex}%
  {1.5ex \@plus.2ex}%
  {\normalfont\normalsize\bfseries}}
\makeatother

%---- User macros (fill these once) ----%
\newcommand{\CourseCodeNumber}{CSC4112}
\newcommand{\ProjectTitle}{Critical Review}
\newcommand{\StudentName}{Navinda Hewawickrama}
\newcommand{\RegistrationNumber}{SC/2020/11730} % <-- change this
\newcommand{\SubmissionDate}{November 2025}
\newcommand{\CoverImagePath}{RUHUNAPNG} 

\begin{document}
%======================%
%  COVER PAGE
%======================%
\begin{titlepage}
  \thispagestyle{empty} 
  \begin{center}
    \vspace*{10mm}
    {\Large \textbf{\CourseCodeNumber}}\\[6mm]
    \includegraphics[width=0.5\textwidth]{\CoverImagePath}\\[10mm]
    
    {\Large \textbf{\ProjectTitle}}\\[10mm]

    % Cover image (optional). Replace width as you like.
    

    \textbf{Registration No.:} \RegistrationNumber\\[2mm]
    \textbf{Student Name:} \StudentName\\[6mm]

    \SubmissionDate\\[12mm]

    \vfill
    \textbf{Bachelor of Computer Science (Special) Degree}\\
    Department of Computer Science, University of Ruhuna
  \end{center}
\end{titlepage}


% start page numbering from here
\clearpage
\pagenumbering{arabic}

%======================%
%  CONTENT OF THE REPORT
%======================%
%--- Chapter 1 ---%
\textbf{Introduction}
The study \enquote{A Next Generation Connectivity Map: L1000 Platform and the First 1,000,000 Profiles}, was pubished in \textbf{Cell}, on November 30th 2017 by Aravind Subramanian, Todd R. Golub, and 56 others. The main purpose of this paper is to show the new scale up version of the \textbf{Connectivity Map (CMap)}, a foundatonal functional look up table that connect genes, drugs, and disease states by virtue of common gene-expression signatures. The 1000, the new, low cost, high throughput reduced  representation expression profiling method, enables generatin of over a million profiles. This is a publication carried out within the NIH Library of Integrated Network-Based Cellular Signatures (LINCS) Consortium which sets the context of the viability and usefulness of developing a genuinely comprehensive, genome-scale functional resource. This analysis will give a composite evaluation of the scale-up transformation brought by the L1000 program and critically assess the methodological limitations of the program, especially genetic perturbation and target recovery rates.

\section{Summary}
\subsection{Introduction}
To truly understand Cellular functions, perturbing the system (genetically or chemically) monitoring downstream consequences, idelly using a \enquote{functional look-up table} is required. Modern bomedical resarch has benefited a lot from genomic resources that provides details about genes and its associations with disease. The original CMap pilot shows this concept, but due to its small scale the usage of it is very limited. By introducing the L1000 dataset, the paper addresses this limitation, a reduced representation approach, to expand CMap dramatically.

\end{document}